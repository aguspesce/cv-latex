\documentclass[a4paper]{letter}
\usepackage[hmargin=2cm, vmargin=1.5cm]{geometry}
\usepackage[utf8]{inputenc}
% \userpackage[spanish]{babel}
% Define font
\usepackage{fontspec}
\setmainfont{Arial}

\usepackage{hyperref}
\hypersetup{
    colorlinks,
    % allcolors=[rgb]{0, 0.451, 0.753},
    allcolors=[rgb]{0.50, 0.00, 0.50},
}
\urlstyle{same}

\usepackage{setspace}
\urlstyle{same}
% Configure spacing and paragraph indentation
% \onehalfspacing
\setlength{\parindent}{3em}

\address{
    % 1-5575 Oka Street\\
    Vancouver, BC, Canada\\
    % Zip code: V6M 2V5
}

\signature{Dr.~Agustina Pesce Lopez \\
    \mbox{\href{mailo:pesce.agustina@gmail.com}{pesce.agustina@gmail.com}} \\
    \href{aguspesce.github.io}{aguspesce.github.io}
}
\newcommand{\coco}{%
    \href{https://www.codecommunicate.org/}{Code to Communicate Program}%
}
\newcommand{\carpentries}{\href{https://carpentries.org/}{The Carpentries}}
\newcommand{\mandyoc}{\href{https://github.com/ggciag/mandyoc}{Mandyoc}}
\newcommand{\xarray}{\href{https://xarray.dev/}{Xarray}}
\newcommand{\matplotlib}{\href{https://matplotlib.org/}{Matplotlib}}
\newcommand{\numpy}{\href{https://numpy.org/}{NumPy}}
\newcommand{\pandas}{\href{https://pandas.pydata.org/}{Pandas}}
\newcommand{\plotly}{\href{https://plotly.com/python/}{Plotly}}
\newcommand{\dash}{\href{https://dash.plotly.com/introduction}{Dash}}
\newcommand{\pytest}{\href{https://docs.pytest.org/en/7.1.x/contents.html}{Pytest}}
\newcommand{\pygmt}{\href{https://www.pygmt.org/latest/}{PyGMT}}
\newcommand{\scipy}{\href{https://scipy.org/}{SciPy}}
\newcommand{\fatiando}{\href{https://www.fatiando.org/}{Fatiando a Terra}}
\newcommand{\jupyter}{\href{https://jupyter.org/}{Jupyter}}
\newcommand{\GoogleCP}{\href{https://cloud.google.com/}{Google Cloud Platform}}
\newcommand{\heroku}{\href{https://www.heroku.com/}{Heroku}}
\newcommand{\geolatinas}{\href{https://geolatinas.org/}{GeoLatinas}}
\newcommand{\geolatinascoding}{
    \href{https://geolatinas.github.io/}{GeoLatinas Coding Group}
}

\newcommand{\position}{Data Scientists}
\newcommand{\company}{Company name}

\begin{document}
    \begin{letter}{}

    \date{\today}

    \opening{\noindent To Whom It May Concern,}

    I am Agustina Pesce Lopez, I would like to apply for the open \position{}
    position.
    This position is an opportunity to combine my passions for data processing
    and visualization, science and programming.
    With my knowledge in physics and my eight years of scientific experience
    managing real dataset, I am confident that I will be a valuable asset to
    your team.

    In the past few years, I have carried out academic research that has
    allowed me to obtain a Physics degree and a PhD in Geophysics.
    During my PhD, I worked in Applied Geophysics within interdisciplinary
    groups to analyze gravity and magnetic data to understand the Earth
    structure.
    This also gave me the experience in data visualization and developing
    workflows to manage and process data sets from diverse sources.

    In this time, I have been able to learn and improve my Python programming
    skills to process, visualize and analyze data.
    I have experience working with \jupyter{} Lab, \numpy{}, \scipy{},
    \matplotlib{}, \plotly{}, \pandas{}, \xarray{}, Python packaging, conda
    environments, among others.
    This also allowed me to familiarize with GitHub workflows, version-control
    systems and best practices for software development.
    In addition, I use Linux as my operating system for a long time, so I feel
    very comfortable working in the command line, either locally or remotely,
    and automating tasks through Bash scripting.

    The last 3 years, I did a postdoc researching subduction zones through
    geodynamical numerical modelling.
    I used to run numerical models that required heavy calculations, so I
    learned to work remotely with Google Cloud Platform and parallelize
    tasks through Open \href{https://www.open-mpi.org/}{MPI}.
    Besides the academic work I did during this period, I have also dedicated
    my time to improve \mandyoc{}, the software we were using within the lab to
    run the numerical simulations.
    I helped to automate the deployment of the documentation to its own website
    through GitHub Actions, I worked on community building (add license, code
    of conduct, how to contribute guidelines, polish the readme) and also
    wrote a Makefile for building and installing the code.
    We proudly achieved to publish it in the
    \href{https://joss.theoj.org/papers/10.21105/joss.04070}{Journal of Open
    Source Software}.

    In addition, I made some contributions in \fatiando{}, which offers
    open-source tools for geophysics.
    I developed some implementation of new features with unit tests and
    documentation.
    I made some improvements of the Fatiando website and I have an active
    participation on Developers and Community meetings.

    I love teaching and sharing my knowledge, so I am working as Coding
    Coordinator and Trainer in \coco, which is a NSF-funded bilingual project
    addressed to Latin and Hispanic students to give them skills in programming
    and science communication.
    In this position, I develop my skills to manager and lead a team of 5+
    trainers to give Python lessons to 20 early career geoscientists.
    In addition, I am part of \geolatinas{} Community.
    This inspiring community is committed to provide technical and
    professional development opportunities to empower and promote
    women/minorities in their careers.
    Therefore, I am part of \geolatinascoding{} where I give Python and Git
    courses and support others member on programming.

    As you will see in my resume, my strong background in scientific research
    has resulted in solid technical skills that align with requirements for
    this position.
    I would be honored if you gave me the opportunity to join your team and
    contribute to \company's strategic goals, applying my strong
    problem-solving and communication skills and learning new tools, to address
    challenges and opportunities this position will offer.

    Please feel free to contact me if you have any questions.
    I have attached my resume with my application.
    Thank you very much for your time and consideration, and I hope to hear
    from you soon.

\closing{Sincerely,}

\end{letter}
\end{document}
