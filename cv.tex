% !TEX TS-program = xelatex
\documentclass[11pt, a4paper]{article}
\usepackage[utf8]{inputenc}
\usepackage[vmargin=2cm, hmargin=2.0cm]{geometry}
\usepackage{enumitem} % fancier lists

% Use Palatino font
\usepackage{fontspec}
% \setmainfont[BoldFont=texgyrepagella-bold.otf, ItalicFont=texgyrepagella-italic.otf, BoldItalicFont=texgyrepagella-bolditalic.otf]{texgyrepagella-regular.otf}

% Define custom environments
\usepackage{environ}

% Disable hyphenation in the CV
% \usepackage[none]{hyphenat}

% Configure paragraph spacings
\setlength{\parindent}{0cm} % Remove indentations

% Configure the font style for sections
\usepackage{titlesec}
\titleformat{\section}
    [hang] % the default shape for sections
    {\normalfont\LARGE\bfseries} % format
    {} % label
    {0pt} % separation (left separation for hang)
    {} % before title
    [\titlerule] % after title
\titleformat{\subsection}
    [hang] % the default shape for sections
    {} % format
    {} % label
    {0pt} % separation (left separation for hang)
    {} % before title

% Disable number of sections. Use this instead of "section*" so that the
% sections still appear as PDF bookmarks. Otherwise, would have to add the
% table of contents entries manually.
\makeatletter
\renewcommand{\@seccntformat}[1]{}
\makeatother

% Configure fancyhdr
\usepackage{fancyhdr}
\usepackage{lastpage} % get the total page numbers (\pageref{LastPage})
\fancyhf{}  % clear current head and footer
\fancyfoot[C]{\small \thepage\ of \pageref{LastPage}}
\renewcommand{\headrulewidth}{0cm} % remove header rule
\pagestyle{fancy}

% Define new colors
\usepackage{xcolor}
\definecolor{mygray}{HTML}{555555}

 % Define multipage tables (used for cv entries)
\usepackage{tabularx}
\usepackage{ltablex}

% Boolean operators
\usepackage{ifthen}

% -----------
% Information
% -----------
\newcommand{\nametitle}{}
\newcommand{\firstname}{Agustina}
\newcommand{\middlename}{}
\newcommand{\familyname}{Pesce Lopez}
\newcommand{\email}{pesce.agustina@gmail.com}
\newcommand{\website}{aguspesce.github.io}
\newcommand{\github}{aguspesce}
\newcommand{\linkedin}{aguspesce}
\newcommand{\orcid}{0000-0002-5538-8845}


% ------------
% Institutions
% ------------
\newcommand{\coco}{
    \href{https://www.codecommunicate.org/}{Code to Communicate Program}
}
\newcommand{\conicet}{
    Consejo Nacional de Investigaciones Científicas y Técnicas
}
\newcommand{\igsv}{Instituto Geofísico Sismológico Volponi}
\newcommand{\fcefn}{
    Facultad de Ciencias Exactas, Físicas y Naturales
}
\newcommand{\unsj}{Universidad Nacional de San Juan}
\newcommand{\fceia}{
    Facultad de Ciencias Exactas, Ingeniería y Agrimensura%
}
\newcommand{\unr}{Universidad Nacional de Rosario}
\newcommand{\carpentry}{\href{https://carpentries.org/}{The Carpentry}}
\newcommand{\mandyoc}{\href{https://github.com/ggciag/mandyoc}{Mandyoc}}
\newcommand{\fatiando}{\href{https://www.fatiando.org/}{Fatiando a Terra}}

% --------------------------
% Variables and environments
% --------------------------
% New variables
\newcommand{\fullname}{\firstname{} \familyname}
% Titles and headings
\newcommand{\maintitle}[1]{
    \begin{center}
        \textbf{\Huge #1}
    \end{center}
}
\newcommand{\subtitle}[1]{
    \begin{center}
        {\large #1}
    \end{center}
}
\newcommand{\affiliation}[1]{
    \begin{center}
        {#1}
    \end{center}
}

% Entries
\newcommand{\entriespad}{0.75em}
\NewEnviron{cventries}{
    \vspace{-1em}
    \begin{tabularx}{\textwidth}{p{0.12\textwidth} p{0.82\textwidth}}
    \BODY
    \end{tabularx}
}
\newcommand{\singleline}[2]{{#1} & {#2} \vspace{\entriespad} \\}
\newcommand{\multiline}[3]{{#1} & {{#2} \newline {#3}} \vspace{0.6em} \\}
% \newcommand{\paper}[5]{
%     {#1} & {{#2}, \emph{#3} \newline {#4} \newline doi:~\DOI{#5}}
%     \vspace{\entriespad} \\
% }
\newcommand{\paper}[5]{{#1} & {{#2}, \emph{#3}} \vspace{\entriespad} \\}
\newcommand{\talk}[5]{
    {#1} & {{#2}, \emph{#3} \ifthenelse{\equal{#4}{}}{}{\newline {#4}}
    \ifthenelse{\equal{#5}{}}{}{\newline doi:~\DOI{#5}}}
    \vspace{\entriespad} \\
}
\newcommand{\education}[3]{{#1} & {\textbf{#2} \newline {#3}} \\}

% Macros
\newcommand{\MAIL}[1]{\href{mailto:#1}{#1}}
\newcommand{\GITHUB}[1]{\href{https://github.com/#1}{@#1}}
\newcommand{\ORCID}[1]{\href{https://orcid.org/#1}{#1}}
\newcommand{\WEBSITE}[1]{\href{https://#1}{#1}}
\newcommand{\DOI}[1]{\href{https://www.doi.org/#1}{#1}}
\newcommand{\LINKEDIN}[1]{\href{https://linkeding.com/#1}{#1}}

% ------------------
% Configure hyperref
% ------------------
\usepackage{hyperref}
\hypersetup{
    colorlinks,
    allcolors=[rgb]{0, 0.451, 0.753},
    pdftitle={\fullname{}'s Curriculum Vitae},
    pdfauthor={\fullname},
}
\urlstyle{same}

% ------------------------------------------------------------------------

\begin{document}


% ------
% Header
% ------
\maintitle{\fullname}
\subtitle{Phd Geophysics | Data Scientist | Physics | Python | GNU/Linux}
% \affiliation{}

\begin{minipage}[t]{0.60\linewidth}
    \begin{flushleft}
        \textbf{Location:} Vancouver, Canada
        \\
        \textbf{email:} \MAIL{\email}
        \\
        \textbf{website:} \WEBSITE{\website}
    \end{flushleft}
\end{minipage}
\hfill
\begin{minipage}[t]{0.40\linewidth}
    \begin{flushright}
        \textbf{ORCID:} \ORCID{\orcid}
        \\
        \textbf{GitHub:} \GITHUB{\github}
        \\
        \textbf{Linkedin:} \LINKEDIN{\linkedin}
    \end{flushright}
\end{minipage}

\color{mygray}{Last modified: \today}

\section{About me}

I have a degree in Physics and a Phd in Geophysics.
I have experience using different Python libraries to process, visualize and
analyze data to solve challenging problems.
This also allowed me to familiarize with GitHub workflows, version-control
systems and best practices for software development.
My background in Physics allow me to insert myself in different topics.
This is complemented with my experience on working with interdisciplinary groups
and remotely.


\section{Education}

\begin{cventries}
    \education{2014 - 2019}{PhD in Geophysics}{\fcefn, \unsj, Argentina.}
    \education{2005 - 2014}{Licentiate in Physics}{\fceia, \unr, Argentina.}
\end{cventries}


\section{Professional Experience}

\begin{cventries}
    \multiline{Nov 2021 - on}{\textbf{Coding Coordinator and Trainer} in
    \coco.}{
    * \emph{Led and supervised} the team of 5+ coding trainers. \newline
    * \emph{Scheduled} weekly meetings with trainers where they reported
    lessons progress.\newline
    * \emph{Organized and supervised} a hackathon where participants developed
    a project using the coding skills learned.\newline
    * \emph{Collaborated} in different tasks to create a good foundation for
    the program.\newline
    * \emph{Created, organised and maintained} a GitHub repository with the
    coding material to teach in the lessons.\newline
    * \emph{Developed} Jupyter notebook with the lesson material about Python
    coding and Git.\newline
    }

    \multiline{Apr 2019 - Mar 2022}{\textbf{Postdoctoral Researcher} in \igsv,
    Argentina.}{
    * \emph{Developed} Python script to create different subduction models and
    plot results using Xarray, NumPy and Matplotlib.\newline
    * \emph{Built} my own Python library to create and plot the models to
    improve and save time in the prossess.\newline
    * \emph{Ran} numerical models using Google Cloud Platform and parallelized
    tasks through Open MPI.\newline
    * \emph{Automated task} using Bash scripting to upload and download data
    from Google Cloud Platform.\newline
    * \emph{Developed and taught} an
    \href{https://github.com/aguspesce/online_seminar_IAG-USP}{Xarray seminary}
    for the lab members.\newline
    * \emph{Presented} the project results in different scientific meetings.
    }

    \multiline{Oct 2019 - Mar 2022}{\textbf{Assistant Professor of Practice} in
    \fcefn, \unsj, Argentina}{
    * \emph{Led, supervised and evaluated} student performance and laboratory
      practices.\newline
    * \emph{Collaborated} in the lesson preparation and participated in the
    teaching lessons.\newline
    * \emph{Taught} physics for 15+ students per week.
    }

    \multiline{Apr 2014 - Mar 2019}{\textbf{PhD Researcher} in \igsv,
    Argentina.}{
    * \emph{Designed, managed and developed} a scientific project archiving a
    PhD Thesis.\newline
    * \emph{Compiled, processed, analyzed and interpreted} gravity and magnetic
    data from different sources (satellite and field) using specialized
    software and different Python libraries (NumPy, Pandas, Matplotlib,
    Cartopy), resulting in some scientific articles and book chapters.\newline
    * \emph{Participated} in the organization of data acquisition in the field
    according to our team data processing needs.\newline
    * \emph{Assisted} peers to improve their publications plotting and analyzing
    their data.\newline
    * \emph{Presented} the PhD thesis results in different scientific meetings.
    \newline
    Thesis title: \href{https://ri.conicet.gov.ar/handle/11336/84580}{Análisis
    geofísico de la Fosa de Loncopué, Neuquén}
    }
\end{cventries}

\section{Project}
\begin{cventries}
    \multiline{Jun 2022 - On}{\textbf{Maintainer} of one of the core lessons of
    \carpentry.}{
        \href{https://datacarpentry.org/python-ecology-lesson-es/}{Análisis y
        visualización de datos usando Python}}

    \multiline{2020}{\href{https://dashboard-covid-ar.herokuapp.com/}{Dashboard}
    to show the evolution of the COVID-19 in each province of Argentina.}{
    * \emph{Designed and developed} the code using Pandas, Plotly and
    Dash.\newline
    * \emph{Loaded, cleaned and processed} the data using Panda.\newline
    *\emph{Created} interactive plots to show the evolution of the number of
    cases in Argentina using Plotly.\newline
    * \emph{Developed} a interactive web application to show the results for
    each province of Argentina.\newline
    * \emph{Deployed} the web app on Heroku.
    }

    \multiline{Apr 2019 - On}{\textbf{Collaborator} in \mandyoc.}{
    * \emph{Automated deployment} of the documentation website through GitHub
    Actions.\newline
    * \emph{Programmed} the tests to check the correct performing of the
    code using Pytest.\newline
    * \emph{Worked} on community building adding license, code of conduct, how
    to contribute guidelines and Readme to improve the repository.\newline
    * \emph{Implemented} the gallery examples using Jupyter notebooks.\newline
    * \emph{Developed} a Makefile for building and installing.\newline
    * \emph{Collaborate} in the publication of \mandyoc code in the
    \href{https://joss.theoj.org/papers/10.21105/joss.04070}{Journal of Open
    Source Software}.
    }

    \multiline{2016 - On}{\textbf{Collaborator} in \fatiando.}{%
    * \emph{Implemented} new methodologies with unit tests using Pytest,
    documentation and use case.\newline
    * \emph{Improved} the main website project.\newline
    * \emph{Create} new examples notebook explaining how to use the
    library.\newline
    * \emph{Fixed} CI errors.\newline
    * \emph{Made} maintenance tasks (CI fixes, automation of tasks, website).\newline
    * \emph{Active participation} on Developers and Community meetings.
    }

    \singleline{2021}{\textbf{Developer} of
        \href{https://dianaceroallard.github.io/}{Diana Acero} personal
        website.}

    \singleline{2021}{\textbf{Developer} of
        \href{https://geolatinas.github.io/}{Geolatinas coding group} website.}

    \singleline{2021}{\textbf{Developer} of
        \href{https://aguspesce.github.io/web-cromografica}{CromoGráfica}
        website. It is a business website currently under development.}{}
\end{cventries}


\section{Technical Skills}

\begin{description}
    \item[Programming:] Python (NumPy, Pandas, Xarray, Matplotlib, Plotly, Dash,
    Pygmt), Numba, bash, FORTRAN, C \item[Markup] Markdown, LaTeX, HTML
    \item[WebDev:] CSS, Bootstrap, Normalize, Static Site Generators (jekyll,
        urubu)
    \item[DevOps:] GNU/Linux, Unix terminal, VIM, Neovim, VS Code, git, GNU Make, SSH
    \item[Graphic Design:] Inkscape, GIMP, Krita
    \item[Other tools:] Jupyter Notebooks, JupyterLAb, LibreOffice Suite, GitHub
        Actions, maxima
\end{description}

\section{Languages}

\begin{description}
    \item[Spanish:] Native
    \item[English:] Advanced
\end{description}

\section{Certifications}

\begin{cventries}
    \singleline{2022}{Maintainer for The Carpentry}
    \singleline{2021}{Certified Software Carpentry Instructor}
\end{cventries}


\section{Awards and Scholarships}

\begin{cventries}
    \singleline{2019 - 2022}{Postdoctoral Scholarship from \conicet.}
    \singleline{2014 - 2019}{Phd scholarship from \conicet.}
    \singleline{2015}{Travel grants: SEG/ExxonMobil Student Education Program
    (SEP), New Orleans, USA.}
\end{cventries}


\section{Last Publications}
\subsection{Peer-reviewed papers}

\begin{cventries}
    \paper{2022}{\href{https://joss.theoj.org/papers/10.21105/joss.04070.pdf}{Mandyoc:
    A finite element code to simulate thermochemical convection in
    parallel}}{Journal of Open-Source Software, 7(71). 4070.}{}

    \paper{2021}{\href{https://revista.geologica.org.ar/raga/article/view/246}{Sección
    eléctrica cortical a través de la fosa de Loncopué}}{Revista de la Asociación
    Geológica Argentina 78 (2), 333-337.}{}

    \paper{2020}{\href{https://doi.org/10.1016/j.tecto.2020.228402}{Oligocene
    to present shallow subduction beneath the southern Puna
    plateau}}{Tectonophysics.}{}{}
\end{cventries}

\subsection{Books Chapters}

\begin{cventries}
    \paper{2020}{\href{https://link.springer.com/chapter/10.1007/978-3-030-29680-3_22}{Pliocene to Quaternary Retroarc Extension in the Neuquén
    Basin: Geophysical Characterization of the Loncopué Trough}}{Opening and
    closure of the Neuquén Basin in the Southern Andes, Springer}{}

    \paper{2020}{\href{https://link.springer.com/chapter/10.1007/978-3-030-29680-3_20}{Plume Subduction Beneath the Neuquén Basin and the Last Mountain Building Stage of the Southern Central Andes}}{Opening and
    closure of the Neuquén Basin in the Southern Andes, Springer}{}

\end{cventries}

\section{Last Conference Proceedings and Talks}

\begin{cventries}
    \talk{2022}{\href{https://www.youtube.com/watch?v=wzrIF4zpshM&feature=emb_title}{Mandyoc:
    A finite element code to simulate thermochemical convection in parallel}}{presented at Transform 2022.}{}{}

    \talk{2021}{\href{https://github.com/GeoLatinas/intro-to-git-2021}{Introduction
    to Git and GitHub}}{for GeoLatinas.}{}{}

    \talk{2021}{\href{https://github.com/fatiando/2021-gsh}{Fatiando a Terra:
    Open-source tools for geophysics}}{Online talk given to the Geophysical
    Society of Houston (GSH).}{}{}

    \talk{2021}{\href{https://doi.org/10.5194/egusphere-egu21-8291}{Harmonica
    and Boule: Modern Python tools for geophysical gravimetry}}{EGU2021 General
    Assembly.}{}{}

    \talk{2020}{\href{https://doi.org/10.5194/egusphere-egu2020-734}{Evaluation
    of the presence of a weak layer in the numerical simulation of lithospheric
    subduction}}{EGU2020 General Assembly.}{}{}
\end{cventries}


\end{document}
