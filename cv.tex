% !TEX TS-program = xelatex
\documentclass[11pt, a4paper]{article}
\usepackage[utf8]{inputenc}
\usepackage[vmargin=2cm, hmargin=2.0cm]{geometry}
\usepackage{enumitem} % fancier lists

% Use Palatino font
\usepackage{fontspec}
\setmainfont{Arial}

% Define custom environments
\usepackage{environ}

% Disable hyphenation in the CV
% \usepackage[none]{hyphenat}

% Configure paragraph spacings
\setlength{\parindent}{0cm} % Remove indentations

% Configure the font style for sections
\usepackage{titlesec}
\titleformat{\section}
    [hang] % the default shape for sections
    {\normalfont\LARGE\bfseries} % format
    {} % label
    {0pt} % separation (left separation for hang)
    {} % before title
    [\titlerule] % after title
\titleformat{\subsection}
    [hang] % the default shape for sections
    {} % format
    {} % label
    {0pt} % separation (left separation for hang)
    {} % before title

% Disable number of sections. Use this instead of "section*" so that the
% sections still appear as PDF bookmarks. Otherwise, would have to add the
% table of contents entries manually.
\makeatletter
\renewcommand{\@seccntformat}[1]{}
\makeatother

% Configure fancyhdr
\usepackage{fancyhdr}
\usepackage{lastpage} % get the total page numbers (\pageref{LastPage})
\fancyhf{}  % clear current head and footer
\fancyfoot[C]{\small \thepage\ of \pageref{LastPage}}
\renewcommand{\headrulewidth}{0cm} % remove header rule
\pagestyle{fancy}

% Define new colors
\usepackage{xcolor}
\definecolor{mygray}{HTML}{555555}

 % Define multipage tables (used for cv entries)
\usepackage{tabularx}
\usepackage{ltablex}

% Boolean operators
\usepackage{ifthen}

% -----------
% Information
% -----------
\newcommand{\nametitle}{}
\newcommand{\firstname}{Agustina}
\newcommand{\middlename}{}
\newcommand{\familyname}{Pesce Lopez}
\newcommand{\email}{pesce.agustina@gmail.com}
\newcommand{\website}{aguspesce.github.io}
\newcommand{\github}{aguspesce}
\newcommand{\linkedin}{aguspesce}
\newcommand{\orcid}{0000-0002-5538-8845}

% ------------
% Institutions
% ------------
\newcommand{\coco}{%
    \href{https://www.codecommunicate.org/}{Code to Communicate Program}%
}
\newcommand{\conicet}{Consejo Nacional de Investigaciones Científicas y Técnicas}
\newcommand{\igsv}{Instituto Geofísico Sismológico Volponi}
\newcommand{\fcefn}{Facultad de Ciencias Exactas, Físicas y Naturales}
\newcommand{\unsj}{Universidad Nacional de San Juan}
\newcommand{\fceia}{Facultad de Ciencias Exactas, Ingeniería y Agrimensura}
\newcommand{\unr}{Universidad Nacional de Rosario}
\newcommand{\carpentry}{\href{https://carpentries.org/}{The Carpentry}}
\newcommand{\mandyoc}{\href{https://github.com/ggciag/mandyoc}{Mandyoc}}
\newcommand{\fatiando}{\href{https://www.fatiando.org/}{Fatiando a Terra}}

% --------------------------
% Variables and environments
% --------------------------
% New variables
\newcommand{\fullname}{\firstname{} \familyname}
% Titles and headings
\newcommand{\maintitle}[1]{
    \begin{center}
        \textbf{\Huge #1}
    \end{center}
}
\newcommand{\subtitle}[1]{
    \begin{center}
        {\large #1}
    \end{center}
}
\newcommand{\affiliation}[1]{
    \begin{center}
        {#1}
    \end{center}
}

% Entries
\newcommand{\entriespad}{0.75em}
\NewEnviron{cventries}{
    \vspace{-1em}
    \begin{tabularx}{\textwidth}{p{0.16\textwidth} p{0.8\textwidth}}
    \BODY
    \end{tabularx}
}
\NewEnviron{lista}{%
    \begin{itemize}[leftmargin=3em]
        \setlength\itemsep{0em}
        \BODY
    \end{itemize}
}
\newcommand{\education}[3]{%
    {#1} & {{\bf \large#2}, {#3}} \vspace{\entriespad} \\}
\newcommand{\experience}[4]{
    \begin{minipage}[t]{0.75\textwidth}
        {{\bf\large #2} \newline {#3}}
    \end{minipage}
    \begin{minipage}[t]{0.25\textwidth}
        \begin{flushright}
        {#1}
        \end{flushright}
    \end{minipage}
    {#4} \vspace{\entriespad}
}
\newcommand{\skill}[2]{{\bf \large #1} & {#2} \vspace{\entriespad} \\}
\newcommand{\singleline}[2]{{#1} & {#2} \vspace{\entriespad} \\}
\newcommand{\paper}[3]{{#1} & {{#2}, \emph{#3}} \vspace{\entriespad} \\}
\newcommand{\talk}[3]{{#1} & {{#2}, \emph{#3}} \vspace{\entriespad} \\}

% Macros
\newcommand{\MAIL}[1]{\href{mailto:#1}{#1}}
\newcommand{\GITHUB}[1]{\href{https://github.com/#1}{@#1}}
\newcommand{\ORCID}[1]{\href{https://orcid.org/#1}{#1}}
\newcommand{\WEBSITE}[1]{\href{https://#1}{#1}}
\newcommand{\DOI}[1]{\href{https://www.doi.org/#1}{#1}}
\newcommand{\LINKEDIN}[1]{\href{https://linkeding.com/#1}{#1}}

% ------------------
% Configure hyperref
% ------------------
\usepackage{hyperref}
\hypersetup{
    colorlinks,
    allcolors=[rgb]{0, 0.451, 0.753},
    pdftitle={\fullname{}'s Curriculum Vitae},
    pdfauthor={\fullname},
}
\urlstyle{same}

% ------------------------------------------------------------------------

\begin{document}
% ------
% Header
% ------
\maintitle{\fullname}
\subtitle{Phd Geophysics | Data Scientist | Physics | Python | GNU/Linux | Open
Source}
% \affiliation{}
\vspace{\entriespad}

\begin{minipage}[t]{0.60\linewidth}
    \begin{flushleft}
        \textbf{Location:} Vancouver, Canada
        \\
        \textbf{email:} \MAIL{\email}
        \\
        \textbf{website:} \WEBSITE{\website}
    \end{flushleft}
\end{minipage}
\hfill
\begin{minipage}[t]{0.40\linewidth}
    \begin{flushright}
        \textbf{GitHub:} \GITHUB{\github}
        \\
        \textbf{Linkedin:} \LINKEDIN{\linkedin}
        \\
        \textbf{ORCID:} \ORCID{\orcid}
    \end{flushright}
\end{minipage}

\vspace{\entriespad}
% \color{mygray}{Last modified: \today}


\section{About me}

I have a degree in Physics and a Phd in Geophysics.
I have experience working on different projects using Python libraries to
process, visualize and analyze data to solve challenging problems.
This also allowed me to familiarize with GitHub workflows, version-control
systems and best practices for software development.
My background in Physics allow me to insert myself in different topics.
This is complemented with my experience on working with interdisciplinary groups
and remotely.


\section{Education}

\begin{cventries}
    \education{2014 - 2019}{PhD in Geophysics}{\fcefn, \unsj, Argentina}
    \education{2005 - 2014}{Licentiate in Physics}{\fceia, \unr, Argentina}
\end{cventries}


\section{Professional Experience}

\experience{Nov 2021 -- on}{Coding Coordinator and Trainer}{\coco}{
    \begin{lista}
        \item \textbf{Led and supervised} the team of 5+ coding trainers.
        \item \textbf{Scheduled} weekly meetings with trainers where they
        reported lessons progress.
        \item \textbf{Organized and supervised} a hackathon where participants
        developed a project using the coding skills learned.
        \item \textbf{Collaborated} in different tasks to create a good
        foundation for the program.
        \item \textbf{Created, organised and maintained} a
        \href{https://github.com/CodeToCommunicate/CoCoLessons}{GitHub repository}
        with the coding material to teach in the lessons.
        \item \textbf{Developed} Jupyter notebook with the lesson material about
        \href{https://github.com/CodeToCommunicate/CoCoLessons}{Python
        coding} and \href{https://github.com/CodeToCommunicate/gitLesson}{Git}.
    \end{lista}
}

\experience{Apr 2019 -- Mar 2022}{Postdoctoral Researcher}{\igsv, Argentina}{
    \begin{lista}
        \item \textbf{Developed} pipeline to create a model, run the
         simulation and plot the results for different subduction scenarios
         using Xarray, NumPy and Matplotlib.
         \item \textbf{Built} a
         \href{https://github.com/aguspesce/tapioca}{python tool} to transform
         and visualize the result after running a simulation to improve the
         plotting and save time in the process.
         \item {\bf Automated task} to update, run and download data from
         Google Cloud Platform.
         \item \textbf{Developed and taught} a
         \href{https://github.com/aguspesce/online_seminar_iag-usp}{xarray
         seminary} for the lab members.
         \item \textbf{Presented} project results in international scientific
         meetings.
    \end{lista}
}

\experience{Oct 2019 - Mar 2022}{Assistant Professor of
Practice}{\fcefn, \unsj, Argentina}{
    \begin{lista}
        \item \textbf{Supervised and evaluated} 15+ students performance and
        laboratory practices per week.
        \item \textbf{Collaborated} in the lesson preparation and participated
        in the teaching lessons.
    \end{lista}
}

\experience{Apr 2014 - Mar 2019}{PhD Researcher}{\igsv, Argentina \newline
Thesis title: \href{https://ri.conicet.gov.ar/handle/11336/84580}{Análisis
geofísico de la Fosa de Loncopué, Neuquén.}}{
    \begin{lista}
        \item \textbf{Designed, managed and developed} a scientific project
        achieving a PhD Thesis.
        \item \textbf{Compiled, processed, analyzed and interpreted} gravity
        and magnetic data from different sources (ground and satellite) using
        specialized software and different Python libraries (NumPy, Pandas,
        Xarray, Fatiando a Terra, Matplotlib, Cartopy), resulting in scientific
        articles and book chapters.
        \item \textbf{Participated} in the organization of data acquisition in
        the field according to data processing needs of our team.
        \item \textbf{Assisted} peers to improve their publications plotting
        and analyzing their data.
        \item \textbf{Presented} the PhD thesis results in international
        scientific meetings.
    \end{lista}
}

\section{Project}

\experience{Jun 2022 - On}{Maintainer of
\href{https://datacarpentry.org/python-ecology-lesson-es/}{Análisis y
visualización de datos usando Python}}{One of the core lessons of \carpentry}{}

\experience{2020}{\href{https://dashboard-covid-ar.herokuapp.com/}{Dashboard}}{Visualization
of evolution of the COVID-19 on each province of Argentina}{
    \begin{lista}
        \item \textbf{Designed and developed} the code using Pandas, Plotly and
        Dash.
        \item  \textbf{Loaded, cleaned and processed} the data using Panda.
        \item \textbf{Created} interactive plots to show the evolution of cases
        in Argentina using Plotly.
        \item \textbf{Developed} a interactive web application to show the
        results for each province of Argentina.
        \item \textbf{Deployed} the web app on Heroku.
    \end{lista}
}

\experience{Apr 2019 - On}{Collaborator in \mandyoc}{Open source tool to
simulate the mantle dynamics}{
    \begin{lista}
        \item \textbf{Automated deployment} of the documentation website
        through GitHub Actions.
        \item \textbf{Programmed} the tests to check the correct performing of
        the code using Pytest.
        \item \textbf{Worked} on community building adding license, code of
        conduct, how to contribute guidelines and Readme to improve the
        repository.
        \item \textbf{Implemented} the gallery examples using Jupyter notebooks.
        \item \textbf{Developed} a Makefile for building and installing.
        \item \textbf{Collaborate} in the publication of \mandyoc code in the
        \href{https://joss.theoj.org/papers/10.21105/joss.04070}{Journal of
        Open Source Software}.
    \end{lista}
}

\experience{2016 - On}{Collaborator in \fatiando}{Open source tools for geophysics}{
    \begin{lista}
        \item \textbf{Implemented} new methodologies with unit tests using
        Pytest, documentation and use case.
        \item \textbf{Improved} the main website project.
        \item \textbf{Create} new examples notebook explaining how to use the
        library.
        \item \textbf{Made} maintenance tasks (CI fixes, automation of tasks,
        website).
        \item \textbf{Active participation} on Developers and Community
        meetings.
    \end{lista}
}

\experience{2021}{Developer of \href{https://dianaceroallard.github.io/}{Diana
Acero}}{A personal website}{}

\experience{2021}{Developer of \href{https://geolatinas.github.io/}{Geolatinas
coding group}}{Organization website}{}

\experience{2021}{Developer of
\href{https://aguspesce.github.io/web-cromografica}{CromoGráfica}}{A business
website currently under development}{}


\section{Technical Skills}

\begin{cventries}
    \skill{Programming}{Python (NumPy, Pandas, Xarray, Matplotlib, Plotly,
    Dash, Pygmt), Numba, bash, FORTRAN, C}
    \skill{Markup}{Markdown, LaTeX, HTML}
    \skill{WebDev}{CSS, Bootstrap, Normalize, Static Site Generators
    (jekyll, urubu)}
    \skill{DevOps}{GNU/Linux, Unix terminal, VIM, Neovim, VS Code, git,
    GNU Make, SSH}
    \skill{Other tools}{Jupyter Notebooks, JupyterLab, LibreOffice Suite,
    GitHub Actions, maxima, Inkscape, GIMP, Krita}
\end{cventries}


\section{Languages}

\begin{cventries}
    \skill{Spanish}{Native}
    \skill{English}{Advanced}
\end{cventries}

\section{Certifications}

\begin{cventries}
    \singleline{2022}{Maintainer for The Carpentry}
    \singleline{2021}{Certified Software Carpentry Instructor}
\end{cventries}


\section{Awards and Scholarships}

\begin{cventries}
    \singleline{2019 - 2022}{Postdoctoral Scholarship from \conicet}
    \singleline{2014 - 2019}{Phd scholarship from \conicet}
    \singleline{2015}{Travel grants: SEG/ExxonMobil Student Education Program
    (SEP), New Orleans, USA}
\end{cventries}


\section{Last Publications}
\subsection{Peer-reviewed papers}

\begin{cventries}
    \paper{2022}{\href{https://joss.theoj.org/papers/10.21105/joss.04070.pdf}{Mandyoc:
    A finite element code to simulate thermochemical convection in
    parallel}}{Journal of Open-Source Software, 7(71). 4070.}

    \paper{2021}{\href{https://revista.geologica.org.ar/raga/article/view/246}{Sección
    eléctrica cortical a través de la fosa de Loncopué}}{Revista de la Asociación
    Geológica Argentina 78 (2), 333-337.}

    \paper{2020}{\href{https://doi.org/10.1016/j.tecto.2020.228402}{Oligocene
    to present shallow subduction beneath the southern Puna
    plateau}}{Tectonophysics.}
\end{cventries}


\subsection{Books Chapters}

\begin{cventries}
    \paper{2020}{\href{https://link.springer.com/chapter/10.1007/978-3-030-29680-3_22}{Pliocene to Quaternary Retroarc Extension in the Neuquén
    Basin: Geophysical Characterization of the Loncopué Trough}}{Opening and
    closure of the Neuquén Basin in the Southern Andes, Springer}

    \paper{2020}{\href{https://link.springer.com/chapter/10.1007/978-3-030-29680-3_20}{Plume Subduction Beneath the Neuquén Basin and the Last Mountain Building Stage of the Southern Central Andes}}{Opening and
    closure of the Neuquén Basin in the Southern Andes, Springer}
\end{cventries}

\section{Last Conference Proceedings and Talks}

\begin{cventries}
    \talk{2022}{\href{https://www.youtube.com/watch?v=wzrIF4zpshM&feature=emb_title}{Mandyoc:
    A finite element code to simulate thermochemical convection in parallel}}{presented at Transform 2022.}

    \talk{2021}{\href{https://github.com/GeoLatinas/intro-to-git-2021}{Introduction
    to Git and GitHub}}{for GeoLatinas.}

    \talk{2021}{\href{https://github.com/fatiando/2021-gsh}{Fatiando a Terra:
    Open-source tools for geophysics}}{Online talk given to the Geophysical
    Society of Houston (GSH).}

    \talk{2021}{\href{https://doi.org/10.5194/egusphere-egu21-8291}{Harmonica
    and Boule: Modern Python tools for geophysical gravimetry}}{EGU2021 General
    Assembly.}

    \talk{2020}{\href{https://doi.org/10.5194/egusphere-egu2020-734}{Evaluation
    of the presence of a weak layer in the numerical simulation of lithospheric
    subduction}}{EGU2020 General Assembly.}
\end{cventries}

\end{document}
