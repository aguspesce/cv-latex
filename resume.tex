% !TEX TS-program = xelatex
\documentclass[5pt, legal]{article}
\usepackage[utf8]{inputenc}
\usepackage[vmargin=2.0cm, hmargin=2.0cm]{geometry}
\usepackage{enumitem} % fancier lists

% Define font
\usepackage{fontspec}
\setmainfont{Lato}

% Define custom environments
\usepackage{environ}

% Disable hyphenation in the CV
% \usepackage[none]{hyphenat}

% Configure paragraph spacings
\setlength{\parindent}{0cm} % Remove indentations

% Configure the font style for sections
\usepackage{titlesec}
\titleformat{\section}
    [hang] % the default shape for sections
    {\normalfont\LARGE\bfseries} % format
    {} % label
    {0pt} % separation (left separation for hang)
    {} % before title
    [\titlerule] % after title
\titleformat{\subsection}
    [hang] % the default shape for sections
    {} % format
    {} % label
    {0pt} % separation (left separation for hang)
    {} % before title

% Disable number of sections. Use this instead of "section*" so that the
% sections still appear as PDF bookmarks. Otherwise, would have to add the
% table of contents entries manually.
\makeatletter
\renewcommand{\@seccntformat}[1]{}
\makeatother

% Configure fancyhdr
\usepackage{fancyhdr}
\usepackage{lastpage} % get the total page numbers (\pageref{LastPage})
\fancyhf{}  % clear current head and footer
\fancyfoot[C]{\small \thepage\ of~\pageref{LastPage}}
\renewcommand{\headrulewidth}{0cm} % remove header rule
\pagestyle{fancy}

% Define new colors
\usepackage{xcolor}
\definecolor{mygray}{HTML}{555555}

 % Define multipage tables (used for cv entries)
\usepackage{tabularx}
\usepackage{ltablex}

% Boolean operators
\usepackage{ifthen}

% -----------
% Information
% -----------
\newcommand{\nametitle}{}
\newcommand{\firstname}{Agustina}
\newcommand{\middlename}{}
\newcommand{\familyname}{Pesce}
\newcommand{\email}{pesce.agustina@gmail.com}
\newcommand{\website}{aguspesce.github.io}
\newcommand{\github}{aguspesce}
\newcommand{\linkedin}{aguspesce}
\newcommand{\orcid}{0000--0002--5538--8845}

% ------------
% Institutions
% ------------
\newcommand{\coco}{%
    \href{https://www.codecommunicate.org/}{Code to Communicate Program}%
}
\newcommand{\conicet}{Consejo Nacional de Investigaciones Científicas y Técnicas}
\newcommand{\igsv}{Instituto Geofísico Sismológico Volponi}
\newcommand{\fcefn}{Facultad de Ciencias Exactas, Físicas y Naturales}
\newcommand{\unsj}{Universidad Nacional de San Juan}
\newcommand{\fceia}{Facultad de Ciencias Exactas, Ingeniería y Agrimensura}
\newcommand{\unr}{Universidad Nacional de Rosario}
\newcommand{\carpentries}{\href{https://carpentries.org/}{The Carpentries}}
\newcommand{\mandyoc}{\href{https://github.com/ggciag/mandyoc}{Mandyoc}}
\newcommand{\xarray}{\href{https://xarray.dev/}{Xarray}}
\newcommand{\matplotlib}{\href{https://matplotlib.org/}{Matplotlib}}
\newcommand{\numpy}{\href{https://numpy.org/}{NumPy}}
\newcommand{\pandas}{\href{https://pandas.pydata.org/}{Pandas}}
\newcommand{\plotly}{\href{https://plotly.com/python/}{Plotly}}
\newcommand{\dash}{\href{https://dash.plotly.com/introduction}{Dash}}
\newcommand{\pytest}{\href{https://docs.pytest.org/en/7.1.x/contents.html}{Pytest}}
\newcommand{\pygmt}{\href{https://www.pygmt.org/latest/}{PyGMT}}
\newcommand{\scipy}{\href{https://scipy.org/}{SciPy}}
\newcommand{\fatiando}{\href{https://www.fatiando.org/}{Fatiando a Terra}}
\newcommand{\jupyter}{\href{https://jupyter.org/}{Jupyter}}
\newcommand{\GoogleCP}{\href{https://cloud.google.com/}{Google Cloud Platform}}
\newcommand{\heroku}{\href{https://www.heroku.com/}{Heroku}}
\newcommand{\geolatinas}{\href{https://geolatinas.org/}{GeoLatinas}}
\newcommand{\scikitlearn}{\href{https://scikit-learn.org/stable/index.html}{scikit-lear}}
\newcommand{\seaborn}{\href{https://seaborn.pydata.org}{seaborn}}
\newcommand{\click}{\href{https://palletsprojects.com/p/click/}{Click}}
\newcommand{\setuptools}{\href{https://setuptools.pypa.io/en/latest/index.html}{setuptools}}

% --------------------------
% Variables and environments
% --------------------------
% New variables
\newcommand{\fullname}{\firstname{} \familyname}
% Titles and headings
\newcommand{\maintitle}[1]{
    \begin{center}
        \textbf{\Huge #1}
    \end{center}
}
\newcommand{\subtitle}[1]{
    \begin{center}
        {\large #1}
    \end{center}
}
\newcommand{\affiliation}[1]{
    \begin{center}
        {#1}
    \end{center}
}

% Entries
\newcommand{\entriespad}{0.75em}
\NewEnviron{cventries}{
    \vspace{-1em}
    \begin{tabularx}{\textwidth}{p{0.15\textwidth} p{0.8\textwidth}}
    \BODY%
    \end{tabularx}
}
\NewEnviron{lista}{%
    \begin{itemize}[leftmargin=3em]
        \setlength\itemsep{0em}
        \BODY%
    \end{itemize}
}
\newcommand{\education}[3]{%
    {#1} & {{\bf \large#2}, {#3}} \vspace{\entriespad} \\}
\newcommand{\experience}[4]{
    \begin{minipage}[t]{0.75\textwidth}
        {{\bf\large #2} \newline {#3}}
    \end{minipage}
    \begin{minipage}[t]{0.25\textwidth}
        \begin{flushright}
        {#1}
        \end{flushright}
    \end{minipage}
    {#4} \vspace{\entriespad}
}
\newcommand{\skill}[2]{{\bf \large #1} & {#2} \vspace{\entriespad}\\}
\newcommand{\singleline}[2]{{#1} & {#2} \vspace{\entriespad}\\}
\newcommand{\paper}[3]{{#1} & {{#2}, \emph{#3}} \vspace{\entriespad}\\}
\newcommand{\talk}[3]{{#1} & {{#2}, \emph{#3}} \vspace{\entriespad}\\}
\newcommand{\service}[2]{{\bf \large #1} \newline {#2} \vspace{\entriespad}\\}

% Macros
\newcommand{\MAIL}[1]{\href{mailto:#1}{#1}}
\newcommand{\GITHUB}[1]{\href{https://github.com/#1}{@#1}}
\newcommand{\ORCID}[1]{\href{https://orcid.org/#1}{#1}}
\newcommand{\WEBSITE}[1]{\href{https://#1}{#1}}
\newcommand{\DOI}[1]{\href{https://www.doi.org/#1}{#1}}
\newcommand{\LINKEDIN}[1]{\href{https://linkedin.com/in/#1}{#1}}


% ------------------
% Configure hyperref
% ------------------
\usepackage{hyperref}
\hypersetup{
    colorlinks,
    % allcolors=[rgb]{0, 0.451, 0.753},
    allcolors=[rgb]{0.50, 0.00, 0.50},
    pdftitle={\fullname{}'s Curriculum Vitae},
    pdfauthor={\fullname},
}
\urlstyle{same}

% ------------------------------------------------------------------------

\begin{document}
% ------
% Header
% ------
\maintitle{\fullname}
% \subtitle{
%     PhD Geophysics | Data Scientist | Physics | Software Developer | Python
%     | GNU/Linux
% }
\affiliation{}
\vspace{\entriespad}

\begin{minipage}[t]{0.60\linewidth}
    \begin{flushleft}
        \textbf{Location:} Vancouver, Canada
        \\
        \textbf{email:} \MAIL{\email}
        \\
        \textbf{website:} \WEBSITE{\website}
        \\
    \end{flushleft}
\end{minipage}
\hfill
\begin{minipage}[t]{0.40\linewidth}
    \begin{flushright}
        \textbf{GitHub:} \GITHUB{\github}
        \\
        \textbf{Linkedin:} \LINKEDIN{\linkedin}
        \\
        \textbf{ORCID:} \ORCID{\orcid}
    \end{flushright}
\end{minipage}

\vspace{\entriespad}
% \color{mygray}{Last modified: \today}

\vspace{\entriespad}

\section{Professional Experience}
\experience{Nov 2021~--~on}{Coding Coordinator and Trainer}{\coco}{%
    \newline
    A NSF-funded bilingual coding and science communication training program
    for early career geoscientists.
    \begin{lista}
        \item {\bf Collaborated} in multiple tasks including {\bf curriculum
            development}, {\bf people management} and updating file records to
            stablish a strong foundation for the 10-week program focused on
            teaching Python and science communication, as well as a 1-week
            hackathon where participants developed a shared project.
        \item {\bf Led} and {\bf supervised} a team of 5+ trainers to teach
            coding skills to 20 students, who proved good coding proficiency
            and communication skills by the end of the program.
        \item {\bf Created} and {\bf maintained} a
            \href{https://github.com/CodeToCommunicate/CoCoLessons}{GitHub
            repository} containing the course material: \jupyter{} Notebooks
            used to teach during each lesson.
            This ensured that the course materials were up-to-date and easily
            accessible for future students.
        % \item {\bf Reported} the progress of the project to superior
        % managers and external evaluators.
    \end{lista}
}
\experience{Apr 2019~--~Mar 2022}{Postdoctoral Researcher}{\igsv, Argentina}{%
    Project title: {\em Influence of a mantle plume in subduction zones by
    geodynamics numerical models.}
    \begin{lista}
        \item {\bf Acquired} the knowledge to operate \mandyoc{}, a software
            for running geodynamical numerical simulations of the Earth's
            interior.
        \item {\bf Developed} a Bash pipeline that enables the creation of
            subduction models for \mandyoc{}, facilitates the remote
            execution of simulation on \GoogleCP{}, and allows for easy
            download of the outputs.
        \item {\bf Built}
            \href{https://github.com/aguspesce/tapioca}{tapioca}, a Python
            package to transform and visualize the outputs of \mandyoc{} using
            \xarray{} and \matplotlib{}.
        \item {\bf Gave} an
            \href{https://github.com/aguspesce/online_seminar_iag-usp}{online
            seminar} to instruct lab members on the handling of
            multidimensional arrays with \xarray{}.
        \item {\bf Presented project results} in
            \href{https://doi.org/10.5194/egusphere-egu2020-734}{international
            scientific meetings}, demonstrating my ability to effectively
            communicate technical information to a wide audience.
    \end{lista}
}
\experience{Oct 2019~--~Mar 2022}{Assistant Professor of Practice}{%
    \fcefn, \unsj, Argentina}{
    \begin{lista}
        \item {\bf Led} the practice and lab classes of Physics courses for
            30+ Geology students.
        \item {\bf Evaluated} students' performance through quizzes, exams and
            laboratory practices.
        \item {\bf Collaborated} in the lesson preparation and {\bf
            participated} in Physics Lectures.
        \item {\bf Set up} and {\bf maintained} online classroom during the
            pandemic and {\bf instructed} other Professors on how to take
            advantage of its tools.
    \end{lista}
}
\experience{Apr 2014~--~Mar 2019}{PhD Researcher}{%
    \igsv, Argentina
    \newline
    Thesis title:
        {\em \href{https://ri.conicet.gov.ar/handle/11336/84580}{%
            Geophysical analysis of the Loncopué Trough, Neuquén, Argentina}}}{%
    \begin{lista}
        \item {\bf Developed} and {\bf further explored} a project throughout a
            5-year PhD which was {\bf funded} by the \conicet{}.
        \item {\bf Compiled} and {\bf preprocessed} gravity and magnetic
            datasets from different sources (ground and satellite) using
            specialized software and Python libraries like \numpy{}, \pandas{},
            \xarray{} and \fatiando{}.
        \item {\bf Applied geophysical processing steps} to produce
            interpretable maps of the study area using \matplotlib{} and
            \pygmt{}.
        \item {\bf Inverted} the gravity data to get better understanding of
            the underlying structures and bodies beneath the Earth's surface
            using \fatiando{} tools.
        \item {\bf Published research results} in peer-reviewed scientific
            journals and {\bf contributed} in the writing of book chapters.
        \item {\bf Presented} my research in international scientific meetings.
        \item {\bf Assisted my peers} to improve their research, achieving
            higher quality scientific publications.
    \end{lista}
}


\section{Projects}
\experience{Jun 2022~--~On}{%
    Maintainer of collaborative Python lesson
    }{%
    \carpentries{}
    }{%
    \begin{lista}
        \item {\bf Assigned} the role of maintainer of
            \href{https://datacarpentry.org/python-ecology-lesson-es/}{Análisis
            y visualización de datos usando Python}, one of the core lessons of
            \carpentries{}.
        \item {\bf Participated} in maintainers' meetings discussing how to
            improve the current version of the lesson.
            % This demonstrated my ability to work collaboratively with a team
            % and my commitment to improving the quality of the lesson.
        \item {\bf Contributed} to the improvement of
            \href{https://github.com/swcarpentry/git-novice-es}{Control de
            versiones con Git} reviewing GitHub Pull Requests.
    \end{lista}
}
\experience{2023}{%
    \href{https://github.com/aguspesce/diabetes_predictor}{Diabetes predictor}
    }{%
    The project goal is to create a model to predict the probability of
    diabetes in patients.
    % This required advanced knowledge of data exploration, visualization, and
    % classification algorithms in Python.
    }{%
    \begin{lista}
        \item {\bf Utilized} \pandas{} and \seaborn{} to explore and visualize
            the data in a \jupyter{} Notebook.
        \item {\bf Created a code} to test 3 different classification
            algorithms, including logistic regression and decision trees, to
            generate a prediction model using the \scikitlearn{} tools.
    \end{lista}
}
\experience{2022}{%
    \href{https://github.com/aguspesce/journal_manager}{Journal manager}}{%
    A custom command‐line tool developed using Python to manage and organize my
    weekly tasks and activity logs across multiple projects.
    }{%
    \begin{lista}
        \item {\bf Designed} and {\bf implemented} the code using \click{}.
        \item {\bf Wrote} a Makefile to simplify the installation process.
    \end{lista}
    }
\experience{2020}{
    \href{https://github.com/aguspesce/dashboard-covid-ar/}{COVID-19 dashboard}
    }{%
    Visualization of the evolution of COVID-19 on each province of Argentina}{
    \begin{lista}
        \item {\bf Loaded}, {\bf cleaned} and {\bf processed} the data using
            \pandas{}.
        \item Utilized \plotly{} and \dash{} to {\bf create interactive plots}
            that allowed users to easily track and analyze the spread of the
            virus across different regions in Argentina.
        % \item {\bf Developed} an interactive web application and
        %     {\bf deployed} it on \heroku{}.
    \end{lista}
}
\experience{Apr 2019~--~Mar 2022}{\mandyoc{} collaborator}{Open source tool to
    simulate the mantle dynamics}{
    \begin{lista}
        \item {\bf Automated the deployment} of the documentation website
            using GitHub Actions, ensuring that users can easily access
            up-to-date documentation.
        \item {\bf Designed} and {\bf coded tests} to check the correct
            performing of the code using \pytest{}.
        \item {\bf Worked on community building} adding license, code of
            conduct, how to contribute guidelines and Readme to improve the
            repository.
        \item {\bf Restructured} the examples gallery using \jupyter{}
            Notebooks to show how to use the code with real examples.
        \item {\bf Developed} a Makefile for building and installing the
            program.
        \item {\bf Collaborate} in the publication of \mandyoc{} code in the
            \href{https://joss.theoj.org/papers/10.21105/joss.04070}{Journal of
            Open Source Software}.
    \end{lista}
}
\experience{2016~--~On}{\fatiando{} collaborator}{Open source tools for geophysics}{%
    \begin{lista}
        \item {\bf Implemented new features} with unit tests using \pytest{},
            documentation and an example of how to use it.
        \item {\bf Improved} the main website project by updating content and
            optimizing performance.
        \item {\bf Created new examples notebooks} explaining how to use the
            library.
        \item {\bf Made maintenance tasks} to fix CI, code automated tasks and
            delete deprecated code.
        \item {\bf Participated} in developers and community meetings to
            discuss how to improve the current tools, cultivate the community,
            design examples, etc.
    \end{lista}
}
\experience{2021}{Website developer}{}{%
    \begin{lista}
    \item {\bf Designed and implemented} responsive website layouts using HTML,
            CSS for different projects:
            \begin{lista}
                \item \href{https://dianaceroallard.github.io/}{%
                    Diana Acero personal website}.
                \item \href{https://geolatinas.github.io/}{%
                    Geolatinas coding group: organization website}
                \item \href{https://aguspesce.github.io/web-cromografica}{%
                    CromoGráfica: business website currently under development}.
            \end{lista}
        \item {\bf Created a clean code} for all projects and {\bf deployed}
            it using GitHub Actions and GitHub Pages.
        \item {\bf Maintained} and {\bf updated} websites based on feedback
            from stakeholders and users.
    \end{lista}
}


\section{Education}
\begin{cventries}
    \education{2014~--~2019}{PhD in Geophysics}{\fcefn, \unsj, Argentina}
    \education{2005~--~2014}{Licentiate in Physics}{\fceia, \unr, Argentina}
\end{cventries}


\section{Certifications}
\begin{cventries}
    \singleline{2022}{Data Science Bootcamp Online from
        \href{https://codigofacilito.com/programas/ciencia-datos}{%
            Código Facilito
            }}
    \singleline{2022}{%
        Database course from
        \href{https://codigofacilito.com/cursos/base-datos-profesional}{%
            Código Facilito
            }}
    \singleline{2021}{Certified Software Carpentry Instructor}
\end{cventries}


\section{Technical Skills}
\begin{cventries}
    \skill{Programming}{%
        Python (\numpy{}, \pandas{}, \scipy{}, \xarray{}, \scikitlearn{},
        \matplotlib{}, \plotly{}, \dash{}, \pygmt{}, \seaborn{}), bash,
        FORTRAN, C, SQL
        }
    \skill{Markup}{Markdown, LaTeX, HTML}
    \skill{WebDev}{%
        CSS, Bootstrap, Normalize, Static Site Generators (jekyll, urubu)
        }
    \skill{DevOps}{%
        GNU/Linux, Unix terminal, VIM, Neovim, VS Code, git, GNU Make, SSH,
        \setuptools{}
        }
    \skill{Office}{LibreOffice Suite, Microsoft Office}
    \skill{Other tools}{%
        Jupyter notebooks, JupyterLab, GitHub Actions, Maxima, Inkscape, GIMP,
        Krita, Docker, Google Cloud Platform, Google Workspace and Google
        Drive,
        }
\end{cventries}


\section{Languages}
\begin{cventries}
    \skill{Spanish}{Native}
    \skill{English}{Advanced}
\end{cventries}


\section{Service Work}
\service{Part of Steering Council in \fatiando{}}{%
     I help plan, organize, and direct the project's operations and programs.
    }
\service{Part of Steering Council in
    \href{https://opensciencelabs.org/}{Open Science Labs}}{%
    I help promote a friendly virtual space for learning open-source tools for
    people in Latin America.
    }
\service{Volunteers in \href{https://2023.pycascades.com/}{PyCascade 2023}}{%
    I collaborated at the Check-in Desk/Registration.
    }
\service{Member of \href{https://www.pyopensci.org/}{pyOpenSci} community}{%
    I have contributed by conducting a peer review for Xclim.
    Additionally, I am helping to create a mentorship program aimed at engaging
    individuals from diverse backgrounds and identities in all aspects of the
    peer review process.
    }
\service{Member of International \geolatinas{} community}{%
     I support the community by giving Python and Git courses, developing the
     coding group website, and mentoring other members on programming.
    }
\service{Member of \carpentries{} community}{%
    I contribute to the community as a lesson maintainer and instructor.
    }
\service{%
    Technical advisor in
    \href{https://academy.climatematch.io/}{Climatematch Academy}.
    }{%
    I am collaborating in the creation of the infrastructure to develop the
    lessons, build the website and solve technical issues.
    }


\section{Awards and Scholarships}
\begin{cventries}
    \singleline{2019~--~2022}{Postdoctoral Scholarship from \conicet{}}
    \singleline{2014~--~2019}{PhD scholarship from \conicet{}}
    \singleline{2015}{Travel grants: SEG/ExxonMobil Student Education Program
        (SEP), New Orleans, USA}
\end{cventries}


\section{Highlight Publications}
\subsection{Peer-reviewed papers}
\begin{cventries}
    \paper{2022}{%
        \href{https://joss.theoj.org/papers/10.21105/joss.04070.pdf}{%
            Mandyoc: A finite element code to simulate thermochemical
            convection in parallel}%
        }{Journal of Open-Source Software, 7(71). 4070.}
    \paper{2021}{%
        \href{https://revista.geologica.org.ar/raga/article/view/246}{%
            Sección eléctrica cortical a través de la fosa de Loncopué}%
        }{Revista de la Asociación Geológica Argentina 78 (2), 333--337.}
    \paper{2020}{%
        \href{https://doi.org/10.1016/j.tecto.2020.228402}{%
        Oligocene to present shallow subduction beneath the southern}}{%
        Tectonophysics.}
\end{cventries}

\subsection{Books Chapters}
\begin{cventries}
    \paper{2020}{%
        \href{https://link.springer.com/chapter/10.1007/978-3-030-29680-3_22}{%
            Pliocene to Quaternary Retroarc Extension in the Neuquén Basin:
            Geophysical Characterization of the Loncopué Trough}%
        }{Opening and closure of the Neuquén Basin in the Southern Andes, Springer}
    \paper{2020}{%
        \href{https://link.springer.com/chapter/10.1007/978-3-030-29680-3_20}{%
            Plume Subduction Beneath the Neuquén Basin and the Last Mountain
            Building Stage of the Southern Central Andes}%
        }{Opening and closure of the Neuquén Basin in the Southern Andes, Springer}
\end{cventries}

\section{Highlight Talks}
\begin{cventries}
    \talk{2022}{%
        \href{https://www.youtube.com/watch?v=wzrIF4zpshM&feature=emb_title}{%
            Mandyoc: A finite element code to simulate thermochemical convection
            in parallel}%
        }{presented at Transform 2022.}
    \talk{2021}{%
        \href{https://github.com/GeoLatinas/intro-to-git-2021}{%
            Introduction to Git and GitHub}%
        }{for GeoLatinas.}
    \talk{2021}{%
        \href{https://github.com/fatiando/2021-gsh}{%
            Fatiando a Terra: Open-source tools for geophysics}%
        }{Online talk given to the Geophysical Society of Houston (GSH).}
    \talk{2021}{%
        \href{https://doi.org/10.5194/egusphere-egu21-8291}{%
            Harmonica and Boule: Modern Python tools for geophysical
            gravimetry}%
        }{EGU2021 General Assembly.}
    \talk{2020}{%
        \href{https://doi.org/10.5194/egusphere-egu2020-734}{%
            Evaluation of the presence of a weak layer in the numerical
            simulation of lithospheric subduction}%
        }{EGU2020 General Assembly.}
\end{cventries}

\end{document}

